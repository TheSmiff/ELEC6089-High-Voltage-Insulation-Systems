%  Pt1.tex
% !TeX spellcheck = en_GB
% !TeX root = ProjectRiskManagement.tex

\section{The PUMP Approach to Uncertainty and Underlying Complexity Management}

%Introduction to risk and opportunity, underlying complexity. 

%Challenge of traditional view of risk.
The phrasing ``uncertainty and underlying complexity management'' has been specifically selected as the title of this section to contrast with the risk management title of the the course as a whole.
Traditional views of risk management offer a very limited scope and an incomplete picture.
Often the focus is on event uncertainty reflecting the standard dictionary definition of risk: \textit{a hazard, chance of bad consequences, exposure to mischance} \citep{OED}.
This approach does not adequately address the whole of the uncertainty effecting the project, and in the worst case can lead to severe mismanagement and failed delivery of project objectives.
The Performance Uncertainty Management Process (PUMP) framework encourages departure from the project-centric approach advocated by best practice, to consider all corporate, operational and planning sources of uncertainty.
This expanded field of view allows the capture of ambiguity uncertainty, inherent variability, systematic uncertainty, as well as event uncertainty.
Using the PUMP framework shifts the focus of risk management to the achievement of opportunity efficiency and risk efficiency, through the vehicle of uncertainty management.

% execution and delivery strategy shaping phase of a project's life cycle
Procedures are a common way to ensure consistency and quality is maintained throughout repeated applications.
A good procedure is often designed to be simple, repeatable and transparent.
However, this cannot be a uniform approach.
Some high complexity, high uncertainty projects require sophisticated, tailored procedures.
The PUMP framework supports this concept through PUMP packs, that is a set of PUMPs tailored to specific projects and project lifecycle stages.
Particularly, this paper focusses on PUMPs within the context of the execution and delivery (E\&D) strategy shaping stage.

%Project life cycle introduction.
A traditional four stage view of the asset/change lifecycle is a useful starting point to consider the scope of a project.
The four stages are conceptualize, planning, E\&D and Utilization.
As explained, effective uncertainty management requires a macro-view of the entire project to capture the whole of the uncertainty.
This leads to an elaboration of the lifecycle to incorporate 12 stages, each emphasizing a different management purpose.
There is discussion as to the usefulness of such high clarity in the lifecycle, but it provokes in-depth consideration of all types of uncertainty \citep{Ward1995145}.
Both views are shown in figure \ref{Figure:Project_Lifecycle}.

\begin{figure}[!h]
  \centering
    \includegraphics[width = \textwidth]{./Figures/ProjectLifecycleDetailedCurve.pdf} 
\caption{Twelve-stage asset/change lifecycle - adapted from \cite{chapman}}
\label{Figure:Project_Lifecycle}
\end{figure}

The planning stage of the traditional lifecycle is expanded to three shaping stages and three governance stages.
The design, operations and termination (DOT) stage aims to derive a strategy for DOT from the corporate strategy developed in the conceptualize stage.
A basic set of design criteria are built, and the objectives of the project are refined. 
Integrated evaluation is important to ensure non-viable projects are halted before large expenditure.
The E\&D strategy shaping stage takes a form more familiar in traditional project management.
The activity and resource requirements are derived from the corporate strategy and the DOT strategy.
This stage considers aspects of the project execution and asks how the asset/change will be delivered?
Schedule derivation takes place in the E\&D stage, including the reconciliation of corporate expectation and real-world plausibility.
Again, integrated evaluation is vital to ensuring only viable projects proceed to later more expensive stages of the lifecycle.
This paper is concerned with PUMPs particularly tailored to this lifecycle stage.



\subsection{PUMP Process Overview}
%Explain concisely in your own words what you believe are the key overall features of a PUMP approach to project risk management in the execution and delivery strategy shaping phase of a project’s lifecycle. Compare these features with the PMI PIMBOK approach or any other form of common practice you are familiar with if you find this helpful, but focus on the PUMP approach. Use examples to illustrate your discussion if you wish, but concentrate on concepts and principles. This will be a largely descriptive summary of your interpretation of the lectures and associated reading. It will demonstrate your grasp of the central core of the unit’s material as a whole, and should be approached with a view to demonstrating this understanding..

The PUMP approach was developed through the assimilation of other industrial risk processes, building upon the best practice approaches from project management bodies.
The SCERT (Synergistic Contingency Evaluation and Review Technique) was developed by Chris Chapman in conjunction with British Petroleum and Acres International management consultants in 1976.
It was first developed for the Magnus offshore oil project in the North Sea. 
The process has now been refined and assimilated into a completed framework suitable for a variety of clarity approaches.

The generic PUMP process is a seven stage iterative cycle as shown in figure \ref{Figure:Project_Lifecycle}. 
A linear `right first time' approach is not a clarity efficient methodology.
A version of Pareto's principle, sometimes called the 80:20 rule \citep{Pareto1992}, empirically states that 20\% of the issues causes 80\% of the problems. 
The first iteration of the PUMP is a high level sweep to identify the key areas of concern.
Subsequent iterations focus on these issues until a sufficient level of clarity is achieved.
This allows the achievement of clarity efficiency, by minimising time spent on unnecessary detail while achieving the required level of understanding.

\begin{figure}[!h]
  \centering
\subfigure[Flowchart Visualisation]{
    \includegraphics[height = 6cm]{./Figures/PUMPgeneric.pdf} 
	\label{Figure:GenericPUMP_Flow}
   } \quad
\subfigure[Gantt Chart Visualisation]{
    \includegraphics[height = 6cm]{./Figures/PUMPgenericGantt.pdf} 
	\label{Figure:GenericPUMP_Gantt}
   }
\caption{The generic PUMP process - adapted from \cite{chapman}}
\label{Figure:GenericPUMP_Both}
\end{figure}

The initiating phase of the generic PUMP is the \textbf{define} phase.
The define phase defines and develops an understanding of the project in a bid to prepare to ask the right questions in subsequent phases.
It features high level context capture and approach development at a strategic level.
There are two key activities in this phase:
%\begin{itemize}
%\item \textit{Consolidate} - gather and integrate relevant existing information,
%\item \textit{Elaborate and Resolve} - fill the gaps in information and resolve conflicting views and inconsistencies.
%\end{itemize}
A useful framework for adequately addressing these issues is the 7W's: where, who, why, what, whichway, wherewithall, when? 
Using the 7W's to consolidate the existing information, then sub-iterating in order to sufficiently define the project and bridge any gaps is a useful method to complete this phase.
The phase is complete when the project deliverables are fit for purpose.

The \textbf{focus} phase involves scoping the level of analysis required during the E\&D shaping lifecycle stage.
The generic PUMP is tailored to the specific requirements of the project at hand as captured from the corporate context.
The aim is to achieve clarity efficiency by clarifying all working assumptions at an early stage.
The phase ends when the scope, strategy and plan for the tailored PUMP process is fit for purpose.
%\begin{figure}[!h]
%  \centering
%    \includegraphics[width = 0.6\textwidth]{./Figures/FocusFlow.pdf} 
%\caption{Focus the process phase - adapted from \cite{chapman}}
%\label{Figure:Project_Lifecycle}
%\end{figure}

The identification of all sources of uncertainty, relevant response options and conditions is undertaken in the \textbf{identify} phase.
This phase has key distinguishing features that are key to achieving clarity efficiency, and is considered in further detail in section \ref{s:Identify}.

Following the identification of sources of uncertainty, responses and conditions, an understanding of the relative importance of each source can be qualitatively established.
The results of the \textbf{structure} phase need to be as simple as possible, but not misleadingly so.
There is a need to deal with underlying complexity, interdependencies and assumptions between sources within this stage.
Often, complexities are ignored in favour of more well understood sources.
Unidentified complexity can have large consequences left unchecked.
This phase contributes to the clarity efficient nature of PUMPs, by allowing the minimising time spent on less important sources.

The clarify \textbf{ownership} phase allocates financial and managerial responsibility for relevant sources of uncertainty.
In reality, all sources are allocated to a party whether explicitly or by default.
The key activities of this stage include developing a contracting strategy, distinguishing ownership and allocating responsibility.
The aim of the phase is a win-win outcome for all contractual parties, so that client and contractor objectives are aligned.

\textbf{Quantify}

The \textbf{evaluate} phase involves synthesizing the results of the quantify phase and assessing the statistical significance of results.
This phase is core the successful application of the PUMP process, and consolidates much of the understanding of uncertainty.
This phase is considered in detail in section \ref{s:Evaluate}.


\subsection{The Clarity Efficient Approach to Opportunity Efficiency}
This section has given a brief overview of a high clarity PUMP for the E\&D strategy shaping phase, except for the identify and evaluate phases considered later.
Several recurring themes throughout all PUMP phases have been illustrated including the pursuit of clarity efficiency, flexibility and a wholistic approach to uncertainty management.
The following sections will consider the Identify and Evaluate phases in further detail.


%1200words.