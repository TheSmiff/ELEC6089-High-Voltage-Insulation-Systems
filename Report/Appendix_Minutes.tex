%  Appendix_Minutes.tex

\section{Meeting Minutes}
\subsection{Meeting 1 - Kick-off Meeting}
\begin{center}
\begin{longtable}{| m{0.2\textwidth} | m{0.6\textwidth} |} \hline
\textbf{Purpose} & ELEC6089 Bushing Design Kick Off Meeting \\ \hline
\textbf{Date and Time} & Thursday 20th February 13:30 \\ \hline
\textbf{Venue} & GDP Lab Zepler Building, Highfield Campus \\ \hline
\textbf{Participants} & TS (Thomas Smith), DA (David Mahmoodi), BH (Brendan Hickman), PF (Patrick Fong)\\ \hline
\textbf{Apologies} &None \\ \hline
\multirow{4}{*}{\textbf{Agenda}} & Review what we understand of the project so far. \\
 & Understand the tasks required. \\ 
 & Agree expectations of work and schedule. \\
 & Agree date and agenda of next meeting. \\ \hline
\end{longtable}
\end{center}

\subsubsection{Minutes of the Meeting}
\begin{center}
\begin{longtable}{| p{0.05\textwidth} |>{\raggedright\arraybackslash}p{0.15\textwidth} | p{0.5\textwidth} |>{\raggedright\arraybackslash}p{0.175\textwidth}|} \hline
\textbf{ID} & \textbf{Subject} & \textbf{Notes and Discussion} & \textbf{Action} \\ \hline
\endhead
1.0	&	Research prior to the meeting	&	BH uploaded the course text to the Facebook working group which has a section on stress control by floating screens. TS uploaded a project from KTH university that had similar guidelines and had a useful description to compound the lecturenotes for the module. All agreed to research the topic further and read these sections by the next meeting	&  \textbf{ALL A1.0}	 \\ \hline
2.0	&	Current understanding of task	&	The group discussed the task at hand. We need to design the bushing using the iterative formulas from the lectures and then build a COMSOL model. The design must be either radial or axial in grading method.	& -	 \\ \hline
3.0 	& 	Work Breakdown &	The group tried to identify the work to complete. This includes research into field design and grading methods, calculating the bushing design, simulating and report writing. None of these tasks can be completed in parallel, and all need the previous in order to complete the task. Hence each member needs to research, and have knowledge of the design and simulation process. It will become clearer who will be assigned responsibility for what shortly. Currently, remain with all needing to complete research & - \\ \hline
4.0	&	Next Meeting	&	First meeting with G. Chen in 2 weeks, Tuesday 4th March. Before then have a first model and have begun verification. Have group Latex template for collaboration, good layout and presentation marks. Use Github. Next meeting on Wednesday 26th. & - \\ \hline

\end{longtable}
\end{center}

\subsubsection{Action List}
\begin{center}
\begin{longtable}{| p{0.05\textwidth} | >{\raggedright\arraybackslash}p{0.15\textwidth} |  p{0.5\textwidth} | >{\raggedright\arraybackslash}p{0.175\textwidth}|} \hline
\textbf{ID} & \textbf{Action} & \textbf{Comments} & \textbf{Status} \\ \hline
\endhead
A1.0	&	Research	&	All to start research. Make notes of all sources. At least reviewed the lecture notes and Kuffel.	& Open 20th Feb \\ \hline	
\end{longtable}
\end{center}

\emph{Next Meeting: 26th Feb 2014, Location \& Time TBA}


\subsection{Meeting 2 - }
\begin{center}
\begin{longtable}{| m{0.2\textwidth} | m{0.6\textwidth} |} \hline
\textbf{Purpose} & ELEC6089 Bushing Design Kick Off Meeting \\ \hline
\textbf{Date and Time} & Thursday 20th February 13:30 \\ \hline
\textbf{Venue} & GDP Lab Zepler Building, Highfield Campus \\ \hline
\textbf{Participants} & TS (Thomas Smith), DA (David Mahmoodi), BH (Brendan Hickman), PF (Patrick Fong)\\ \hline
\textbf{Apologies} &None \\ \hline
\multirow{4}{*}{\textbf{Agenda}} & Review what we understand of the project so far. \\
 & Understand the tasks required. \\ 
 & Agree expectations of work and schedule. \\
 & Agree date and agenda of next meeting. \\ \hline
\end{longtable}
\end{center}

\subsubsection{Minutes of the Meeting}
\begin{center}
\begin{longtable}{| p{0.05\textwidth} |>{\raggedright\arraybackslash}p{0.15\textwidth} | p{0.5\textwidth} |>{\raggedright\arraybackslash}p{0.175\textwidth}|} \hline
\textbf{ID} & \textbf{Subject} & \textbf{Notes and Discussion} & \textbf{Action} \\ \hline
\endhead
1.0	&	Research prior to the meeting	&	BH uploaded the course text to the Facebook working group which has a section on stress control by floating screens. TS uploaded a project from KTH university that had similar guidelines and had a useful description to compound the lecturenotes for the module. All agreed to research the topic further and read these sections by the next meeting	&  \textbf{ALL A1.0}	 \\ \hline
2.0	&	Current understanding of task	&	The group discussed the task at hand. We need to design the bushing using the iterative formulas from the lectures and then build a COMSOL model. The design must be either radial or axial in grading method.	& -	 \\ \hline
3.0 	& 	Work Breakdown &	The group tried to identify the work to complete. This includes research into field design and grading methods, calculating the bushing design, simulating and report writing. None of these tasks can be completed in parallel, and all need the previous in order to complete the task. Hence each member needs to research, and have knowledge of the design and simulation process. It will become clearer who will be assigned responsibility for what shortly. Currently, remain with all needing to complete research & - \\ \hline
4.0	&	Next Meeting	&	First meeting with G. Chen in 2 weeks, Tuesday 4th March. Before then have a first model and have begun verification. Have group Latex template for collaboration, good layout and presentation marks. Use Github. Next meeting on Wednesday 26th. & - \\ \hline

\end{longtable}
\end{center}

\subsubsection{Action List}
\begin{center}
\begin{longtable}{| p{0.05\textwidth} | >{\raggedright\arraybackslash}p{0.15\textwidth} |  p{0.5\textwidth} | >{\raggedright\arraybackslash}p{0.175\textwidth}|} \hline
\textbf{ID} & \textbf{Action} & \textbf{Comments} & \textbf{Status} \\ \hline
\endhead
A1.0	&	Research	&	All to start research. Make notes of all sources. At least reviewed the lecture notes and Kuffel.	& Open 20th Feb \\ \hline	
\end{longtable}
\end{center}

\emph{Next Meeting: 26th Feb 2014, Location \& Time TBA}