%  Field_Modelling.tex
% !TeX spellcheck = en_GB
% !TeX root = ReportMain.tex

\section{Modelling Results}
The following simulations were completed using the COMSOL multiphysics software package.
COMSOL is a professional finite element simulation package able to model a variety of physical features.
The following models are created using the AC/DC module, which is used to simulate electric and magnetic fields \cite{ComsolACDC}.
Specifically, the electrostatics interface is used. 
This solves a charge conservation equation for a given voltage and spacial distribution of charge \cite{ComsolACDC}.

\subsection{Finite Element Methods (FEM)}
There are inherent difficulties in solving the partial differential equations that govern many practical engineering problems \cite{kuffel2000high}.
Despite knowing the equations and appropriate boundary conditions that govern a problem, many are complicated by irregular geometries or other discontinuities.
Numerical methods allow approximate solutions to be obtained for problems intractable by analytic methods \cite{meshkatoddini2006study}.
In an analytic solution, the whole system is governed by a mathematical equation valid for the entire region of interest. 
Although these differential equations are often mathematically compact, it is difficult to obtain an answer unless the system is unreasonably simplified \cite{meshkatoddini2006study}.
In FEMs, the complex geometry is broken into a series of much smaller and simpler geometries \cite{kuffel2000high}.
These geometries can be squares, rectangles or triangles in 2D or the 3D equivalent shapes. 
These simpler shapes form interconnected subregions for which an approximate function, usually a high order polynomial, can be used to represent the actual function.
If the complex is split into an adequate number of simple shapes, these approximate functions closely matches the exact solution \cite{meshkatoddini2006study}.

By default COMSOL uses a triangular discretisation to split up a complex geometry in a process called meshing.
This forms an unstructured grid of triangles, allowing the mapping of complex or curved geometries.
Other numerical methods such as Finite Difference Methods require a structured grid, hence FEMs are more flexible with regards to geometry \cite{kuffel2000high}.
Meshing requires an initial understanding of the expected outcomes of the problem, so that the mesh can be refined in areas of interest.
Each triangular element is approximated by a linear interpolation of the potential at the vertices of the triangle.
A set of linear algebraic equations are formed by minimising the error between the actual solution and a set of approximate linear trial functions \cite{meshkatoddini2006study}.

\subsection{Equation Derivation}
The electrostatics interface of the AC/DC COMSOL module uses the electric potential $V$ to calculate static electric fields.
A Poisson type partial differential equation is derived using classical electrostatics and Gauss' Law \cite{hayt2012engineering}.

By taking Gauss's Law:
\begin{equation}
\nabla.\mathbf{D} = \rho v
\end{equation}
the equation for electric flux density $\mathbf{D}$:
\begin{equation}
\mathbf{D} = \epsilon_0\epsilon_r\mathbf{E}
\end{equation}
this can be combined with the equation for a static electric field:
\begin{equation}
\mathbf{E} = -\nabla V
\end{equation}
to give by substitution:
\begin{equation}
\nabla.\mathbf{D} = \nabla.(\epsilon_0 \epsilon_r \mathbf{E}) = -\nabla.(\epsilon_0 \epsilon_r \nabla V) = \rho v
\end{equation}
which is more usually written:
\begin{equation}
\nabla^{2}V = -\frac{\rho v}{\epsilon_0 \epsilon_r}
\end{equation}
where $\epsilon_0$ is the permittivity of free space, $\epsilon_r$ is the relative permittivity of the material, $\mathbf{E}$ is the electric field strength and $\rho v$ is the volume charge density.

In the special case where there is zero volume charge density, that is $\rho v = 0$ then the equation simplifies to Laplace's Equation:
\begin{equation}
\nabla^{2}V = 0
\end{equation}

The models used in this paper are 2D axisymmetric, meaning that a 2D model is used to describe a 3D object that can be rotated $360^o$ about a central point $r=0$ to give a 3D geometry.
This assumes that not only is the geometry the same in the $\varphi$ direction, but also that the electric potential is constant.
In this case, Poissons equation can be rewritten in cylindrical coordinates for a 2D axisymmetric model, it is multiplied by r to ensure there are no singularities at $r=0$ \cite{meshkatoddini2006study}.
\begin{equation}
\begin{bmatrix} 
\frac{\partial}{\partial r} \\
\frac{\partial}{\partial z}
\end{bmatrix}^T
.(r \begin{bmatrix}
\frac{\partial V}{\partial r} \\
\frac{\partial V}{\partial z}
\end{bmatrix})
= -\frac{r\rho v}{\epsilon_0 \epsilon_r}
\end{equation}

\inote{Boundary Conditions}

\subsection{Workflow}
In order to simulate the electric field distribution within our bushing design, 2D axisymmetric models were created. The general workflow to achieve this is:
\begin{enumerate}
\item Build a geometry representing the physical structure of the bushing.
\item Assign each geometric domain a material. The material selection determines the relative permittivity $\epsilon_r$ of each domain.
\item Define the charge conservation equation and all initial conditions. This includes setting which boundaries are at ground and conductor potential and setting boundary conditions.
\item Design a mesh. The geometry is split into smaller elements in order to compute the charge conservation equation. For designs with foils, special meshing parameters are required to speed up the process.
\item Carry out the study. This stage is the actual computation of the solution.
\item Post-processing - Display the results in a number of formats including 3D, 2D and 1D plots, or export the data for post-processing in Matlab.
\end{enumerate}


\subsection{Baseline Model}
In order to minimise the computation time required for each model, it was necessary to determine the areas of interest in the model.
A bushing geometry was built with no foils inserted, a high quality mesh was produced and the system was solved to find the electric field distribution throughout the bushing and the surrounding area.
\begin{figure}[!htb]
  \centering
  \subfigure[Geometry]{
    \includegraphics[height = 10cm]{./Figures/Simulations/Edited_No_Foils_Large/Geometry.png} 
	\label{Figure:No_Foil_Large_Geom}
  }
  \subfigure[Extra Fine Mesh]{
    \includegraphics[height = 10cm]{./Figures/Simulations/Edited_No_Foils_Large/Meshing.png} 
	\label{Figure:No_Foil_Large_Mesh}
  }
\subfigure[Normal Electric Field (V/m)]{
    \includegraphics[height = 10cm]{./Figures/Simulations/Edited_No_Foils_Large/Norm_E_Field.png} 
	\label{Figure:No_Foil_Large_Field}
  }
\caption{Baseline Model Simulation - X and Y axis are dimensions in mm}
  \label{Figure:No_Foil_Large}
\end{figure}

The model is made up with a very large geometry. 
The air and oil extends radially 2m from the end of the bushing and 1m in the axial direction.
This is to understand the anticipated area of interest in the model.
By considering figure \ref{Figure:No_Foil_Large_Field} it is clear that there is very little happening further than 500mm radially from the bushing surface and there is very little of interest further than 200mm in the radial direction.
Therefore all further models will adhere to this geometry, ensuring that the area of interest is captured, while decreasing simulation times to a minimum.

\subsection{No Foils}
\dots

\subsection{No Grading}
\begin{figure}[!htb]
  \centering
  \subfigure[Geometry]{
    \includegraphics[height = 15cm]{./Figures/Simulations/Edited_No_Grading_Final/Geometry.png} 
	\label{Figure:No_Foil_Large_Geom}
  }
  \subfigure[Meshing]{
    \includegraphics[height = 15cm]{./Figures/Simulations/Edited_No_Grading_Final/Meshing.png} 
	\label{Figure:No_Foil_Large_Mesh}
  }
\subfigure[Normal Electric Field (V/m)]{
    \includegraphics[height = 15cm]{./Figures/Simulations/Edited_No_Grading_Final/E_Field_Norm.png} 
	\label{Figure:No_Foil_Large_Field}
  }
\caption{Baseline Model Simulation - X and Y axis are dimensions in mm}
  \label{Figure:No_Foil_Large}
\end{figure}

\subsection{Radial Grading}
\dots

\subsection{Axial Grading}
\dots
 

%\subsection{No Grading}
%As a baseline for comparison, a bushing with no foils has been constructed and simulated.
%The geometry of the model was built as in figure \ref{figure:Geom:Nograde}.
%The system is an axialsymmetric 2D model, which takes the central vertical point $r=0$ as the centre of a cylinder.
%\begin{figure}[!h]
%   \centering
%   \includegraphics[width = 0.8\textwidth]{NoGradingBlock.pdf}
%   \caption{COMSOL Geometry Annotated with Materials - No Grading}
%   \label{figure:Geom:Nograde}
%\end{figure}
%
%Once the geometry of the model is defined, a finite element mesh can be created as shown in figure \ref{figure:Mesh:Nograde}.
%This model is fairly simple, hence a very fine graded mesh was used improving the accuracy of results.
%\begin{figure}[!h]
%   \centering
%   \includegraphics[width = 0.8\textwidth]{NoGradingMesh.pdf}
%   \caption{COMSOL Mesh - No Grading}
%   \label{figure:Mesh:Nograde}
%\end{figure}
%
%The next stage is to define the relative permittivity of each of the materials used for each sub section of the geometry.
%The initial conditions must then be set, with the conductor set to 275kV, and the transformer wall and all outer boundaries earthed.
%All other boundaries are assumed to be continuity boundaries.
%
%The model can then be solved to give the electric field distribution
%\inote{TS - Report done up to here 03/03/2014}
.
%\begin{figure}[!h]
%   \centering
%   \includegraphics[width = 0.8\textwidth]{WideNoGrading.png}
%\end{figure}
%
%\begin{figure}[!h]
%   \centering
%   \includegraphics[width = 0.8\textwidth]{CloseNoGrading.png}
%\end{figure}
%
%\begin{figure}[!h]
%   \centering
%   \includegraphics[width = 0.8\textwidth]{SurfaceGraded21.png}
%\end{figure}
%
%\begin{figure}[!h]
%   \centering
%   \includegraphics[width = 0.8\textwidth]{WideGraded21.png}
%\end{figure}
%
%\begin{figure}[!h]
%   \centering
%   \includegraphics[width = 0.8\textwidth]{CloseGraded21.png}
%\end{figure}
%
%\begin{figure}[!h]
%   \centering
%   \includegraphics[width = 0.8\textwidth]{SurfaceGradedCloseish21.png}
%\end{figure}