\section{Electrical Breakdown} %Possibly a better heading needed
There are some consequences that need to be taken into account when designing system in high voltage. High voltage systems have much higher electric potential to their surrounding objects which are usually at earth potential, these arise electric potential gradient or electric field between the systems and their surrounding objects. The values of electric field can be extremely high due to the high voltage and may cause electrical breakdown and partial discharge which can lead to failure in systems. It is important to design system to minimise the chance of these events.
\begin{figure}[!h]
   \centering
   \includegraphics[width = 0.5\textwidth]{lichtenberg-figures004.jpg}
   \caption{Electrical tree tracing the path of damaged insulation caused by electrical breakdown}
   \label{figure:breakdown}
\end{figure}
%http://damnamazing.blogspot.co.uk/2008/11/captured-lightning-lichtenberg-figures.html
Electrical breakdown is the action of electrical conduction across a insulating medium, usually dielectric, when the voltage across this medium exceeds the breakdown voltage subject to the type of material of the insulation. This usually happens when the potential difference is extremely high and arcing can be seen in gaseous insulating medium. This can cause changes to the compounds in the insulating medium and also damage the system's equipments as shown in figure \ref{figure:breakdown}. For example, breakdown in oil insulation due to excessive electric field.

There are different types of electrical breakdown. The main types of electrical breakdown that are involved in high voltage insulation systems are corona discharge in air and partial discharge in dielectric insulation.

\subsection{Corona Discharge}

\subsection{Partial Discharge}
Partial discharge is the electrical breakdown in small portion of dielectric material due to imperfection in the material. The most common cause of partial discharge is void with in the dielectric material, where these void contains material of a lower electrical breakdown strength then the dielectric material. Under high voltage applied across the dielectric insulation, these voids will experience local electrical breakdown if the corona inception voltage is exceeded, another words partial discharge in the dielectric insulation.
