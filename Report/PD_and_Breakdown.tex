\section{Electrical Breakdown in High Voltage Systems} %Possibly a better heading needed
There are some consequences that need to be taken into account when designing systems for operation in high voltage. High voltage systems generate a higher electric potential to their surrounding objects which are usually at earth potential, these arise large electric potential gradients or electric fields. The values of high field regions within the electric field may cause electrical breakdown and partial discharges leading to failure in the system \citep{kuffel2000high}. It is therefore important to design systems to minimise the chance of these events.
\begin{figure}[!h]
   \centering
   \includegraphics[width = 0.5\textwidth]{lichtenberg-figures004.jpg}
   \caption{Electrical tree tracing the path of damaged insulation caused by electrical breakdown}
   \label{figure:breakdown}
\end{figure}
%http://damnamazing.blogspot.co.uk/2008/11/captured-lightning-lichtenberg-figures.html
Electrical breakdown is the action of electrical conduction across an insulating medium, usually dielectric, when the voltage across this medium exceeds the breakdown voltage subject to the type of material of the insulation. This usually happens when the potential difference is extremely high and arcing can be seen in gaseous insulating mediums. This can cause changes to the compounds in the insulating medium and also cause damage to equipment in the form of treeing as shown in figure \ref{figure:breakdown}. 

There are different types of electrical breakdown to consider. The main types of electrical breakdown that are involved in high voltage insulation systems are general breakdown in the system, surface flashover, partial discharge in the dielectric insulation as well as corona discharge in air.

\subsection{General Breakdown} %intrinsic? - cite kuffel
In the case of high voltage transformer bushing, there is a combination of both high electric fields and a close proximity to the grounded surroundings. Clearly then, bushing is designed to insulate the conductor from ground. The most basic form of bushing is non-condenser bushing \cite{HVEngandTesting}. It is the radial component of the tangential electric field that causes this typical breakdown directly from the conductor to the grounded flange. Due to the high voltage requirements, insulation must have a very high dielectric constant; typical materials used are Resin-impregnated paper (RIP) and Oil-impregnated paper (OIP). The importance of this bushing is that as the electric strength of the insulation increases, radial thickness may be reduced \cite{HVEngandTesting}. Typically the general breakdown strength of bushings is very high relative to PD inception and is a secondary concern.

\subsection{Surface Flashover}
The partnering electric breakdown effect for solid, non-condenser bushing is surface flashover; breakdown caused by the electric field travelling from the conductor to the surroundings via the surface of the bushing. The properties governing the axial height of the bushing are both the axial electric field and the surrounding medium \cite{HVEngandTesting}. Typically the criteria for surface flashover breakdown is much lower than general breakdown, and improvements to the bushing have primarily been in shaping the axial field distribution. This investigation considers an air-to-oil bushing and as oil is more than twice as strong dielectrically as air at atmospheric pressure the air end must be approximately twice as long \cite{harlow2004electric}. This 'creepage length' is, in industrial applications, increased by the use of ceramic shedding which will not be investigated in this case study. However it is important to note that it will typically increase the effective axial length of the bushing by a factor of 4.

\subsection{Partial Discharge}
Partial Discharge (PD) is defined by the British Standards 60270:2001 as a “localized electrical discharge that only partially bridges the insulation between conductors and which can, or can not, occur adjacent to a conductor” and as “a consequence of local electrical stress concentrations in the insulation” \cite{60270}. The most common cause of partial discharge is a void, originating from manufacturing imperfections, within the dielectric material where the void contains material of a lower electrical breakdown strength (gas or air) than the dielectric material. Under high electric fields across the dielectric insulation these voids will experience local electrical breakdown as the rate of charge increase is greater than decay and inception voltage is exceeded. Due to these conditions, PDs typically occur under intense tangential electric field and electron emission results \cite{surfaceflashover} and as the discharges spread across the surface of the solid dielectric breakdown will occur \cite{kuffel2000high}. In order to ensure the insulation system can sufficiently reduce the chance of PD, the inception voltage of the insulation must be relatively high compare to the operating voltage of the conductor.

Partial discharge is a primary cause of ageing within High Voltage systems, this is because each successive discharge applies electro-mechanical forces to the void itself and the insulation progressively deteriorates \cite{PDageing}. Therefore reducing tangential electric fields and measurement of PDs is of high importance. 

\subsection{Corona Discharge}
Corona discharge is the ionisation of fluid around a conductor of high electric potential. The ionisation of the surrounding fluid is due to the discharge from the conductor. In high voltage systems, corona discharge occurs most commonly from the conductor to air surrounding the conductor as illustrated in figure \ref{figure:corona}. Corona discharge occurs at area of intense electric field, but not sufficiently intense to cause arcing. This type of breakdown might not cause fatal damage to the insulation, but would shorten the life time of the insulation system. It also represents power losses to the surrounding, so it is important to design insulation system minimising the effect of corona discharge.

\begin{figure}[!h]
   \centering
   \includegraphics[width = 0.5\textwidth]{coronadischarge.jpg}
   \caption{Corona discharge appearing on the insulation surface of high voltage conductor}
   \label{figure:corona}
\end{figure}
%Problem, this picture is from wikipedia