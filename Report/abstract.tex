The complexities involved in working with high voltages is of great importance to bushing design. Transformer bushings are used to insulate high voltage conductors as they pass through the grounded exterior of the transformer. Bushings are required to avoid various electrical breakdown mechanisms caused by the potential difference from high potential equipment to the surrounding earth objects. This report presents a comprehensive investigation of transformer bushing design, considering the different types of breakdown in high voltage systems, varying stress control methods for DC systems and different insulation material. 

Beyond this, the bulk of the report is a comparison of different simulation model’s effectiveness. Solid bushing and capacitive grading bushing, the act of inserting conducting foils within the insulation; non-graded, radial and axial bushing were all simulated and the resulting fields tested for many different failure criteria. Axial grading, by controlling the axial field responsible for surface flashover, was the most effective model. A further effort was made to improve the foil interfaces by attaching semiconductor material to the tip as well as smoothing the foil edge profile. These improved peak field values considerably for the radial design; proof that within bushing design there is still necessary research to be done.  
